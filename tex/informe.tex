\documentclass[a4paper, 10pt, twoside]{article}

\usepackage[top=1in, bottom=1in, left=1in, right=1in]{geometry}
\usepackage[utf8]{inputenc}
\usepackage[spanish, es-ucroman, es-noquoting]{babel}
\usepackage{setspace}
\usepackage{multicol}
\usepackage{fancyhdr}
\usepackage{lastpage}
\usepackage{amsmath}
\usepackage{amsfonts}
\usepackage{amsthm}
\usepackage{verbatim}
\usepackage{graphicx}
\usepackage{float}
\usepackage{enumitem} % Provee macro \setlist
\usepackage{tabularx}
\usepackage{multirow}
\usepackage{hyperref}
\usepackage{multicol}
\usepackage[toc, page]{appendix}


%%%%%%%%%% Configuración de Fancyhdr - Inicio %%%%%%%%%%
\pagestyle{fancy}
\thispagestyle{fancy}
\lhead{Trabajo Práctico 1 · Teoría de las Comunicaciones}
\rhead{Delgado · Lovisolo · Petaccio}
\renewcommand{\footrulewidth}{0.4pt}
\cfoot{\thepage /\pageref{LastPage}}

\fancypagestyle{caratula} {
   \fancyhf{}
   \cfoot{\thepage /\pageref{LastPage}}
   \renewcommand{\headrulewidth}{0pt}
   \renewcommand{\footrulewidth}{0pt}
}
%%%%%%%%%% Configuración de Fancyhdr - Fin %%%%%%%%%%


%%%%%%%%%% Miscelánea - Inicio %%%%%%%%%%
% Evita que el documento se estire verticalmente para ocupar el espacio vacío
% en cada página.
\raggedbottom

% Deshabilita sangría en la primer línea de un párrafo.
\setlength{\parindent}{0em}

% Separación entre párrafos.
\setlength{\parskip}{0.5em}

% Separación entre elementos de listas.
\setlist{itemsep=0.5em}

% Asigna la traducción de la palabra 'Appendices'.
\renewcommand{\appendixtocname}{Apéndices}
\renewcommand{\appendixpagename}{Apéndices}
%%%%%%%%%% Miscelánea - Fin %%%%%%%%%%


%%%%%%%%%% Insertar grafo - Inicio %%%%%%%%%%
\newcommand{\grafo}[1]{
  \includegraphics[type=png,ext=.png,read=.png,width=16cm]{#1}
}
%%%%%%%%%% Insertar grafo - Fin %%%%%%%%%%


\begin{document}


%%%%%%%%%%%%%%%%%%%%%%%%%%%%%%%%%%%%%%%%%%%%%%%%%%%%%%%%%%%%%%%%%%%%%%%%%%%%%%%
%% Carátula                                                                  %%
%%%%%%%%%%%%%%%%%%%%%%%%%%%%%%%%%%%%%%%%%%%%%%%%%%%%%%%%%%%%%%%%%%%%%%%%%%%%%%%


\thispagestyle{caratula}

\begin{center}

\includegraphics[height=2cm]{DC.png} 
\hfill
\includegraphics[height=2cm]{UBA.jpg} 

\vspace{2cm}

Departamento de Computación,\\
Facultad de Ciencias Exactas y Naturales,\\
Universidad de Buenos Aires

\vspace{4cm}

\begin{Huge}
Trabajo Práctico 1
\end{Huge}

\vspace{0.5cm}

\begin{Large}
Teoría de las Comunicaciones
\end{Large}

\vspace{1cm}

Primer Cuatrimestre de 2014

\vspace{4cm}

\begin{tabular}{|c|c|c|}
\hline
Apellido y Nombre & LU & E-mail\\
\hline
Delgado, Alejandro N.  & 601/11 & nahueldelgado@gmail.com\\
Lovisolo, Leandro      & 645/11 & leandro@leandro.me\\
Petaccio, Lautaro José & 443/11 & lausuper@gmail.com\\
\hline
\end{tabular}

\end{center}

\newpage


%%%%%%%%%%%%%%%%%%%%%%%%%%%%%%%%%%%%%%%%%%%%%%%%%%%%%%%%%%%%%%%%%%%%%%%%%%%%%%%
%% Introducción                                                              %%
%%%%%%%%%%%%%%%%%%%%%%%%%%%%%%%%%%%%%%%%%%%%%%%%%%%%%%%%%%%%%%%%%%%%%%%%%%%%%%%


\section{Introducción}


%%%%%%%%%%%%%%%%%%%%%%%%%%%%%%%%%%%%%%%%%%%%%%%%%%%%%%%%%%%%%%%%%%%%%%%%%%%%%%%
%% Desarrollo                                                                %%
%%%%%%%%%%%%%%%%%%%%%%%%%%%%%%%%%%%%%%%%%%%%%%%%%%%%%%%%%%%%%%%%%%%%%%%%%%%%%%%


\section{Desarrollo}



%%%%%%%%%%%%%%%%%%%%%%%%%%%%%%%%%%%%%%%%%%%%%%%%%%%%%%%%%%%%%%%%%%%%%%%%%%%%%%%
%% Resultados                                                                %%
%%%%%%%%%%%%%%%%%%%%%%%%%%%%%%%%%%%%%%%%%%%%%%%%%%%%%%%%%%%%%%%%%%%%%%%%%%%%%%%


\section{Resultados}


\subsection{Red \emph{Alto Palermo}}

\grafo{altopalermo}


\subsubsection{Direcciones IP origen de paquetes ARP \emph{who-has} modeladas como fuentes de información}

\include{altopalermo-source}


\subsubsection{Direcciones IP destino de paquetes ARP \emph{who-has} modeladas como fuentes de información}
% GNUPLOT: LaTeX picture with Postscript
\begingroup
  \makeatletter
  \providecommand\color[2][]{%
    \GenericError{(gnuplot) \space\space\space\@spaces}{%
      Package color not loaded in conjunction with
      terminal option `colourtext'%
    }{See the gnuplot documentation for explanation.%
    }{Either use 'blacktext' in gnuplot or load the package
      color.sty in LaTeX.}%
    \renewcommand\color[2][]{}%
  }%
  \providecommand\includegraphics[2][]{%
    \GenericError{(gnuplot) \space\space\space\@spaces}{%
      Package graphicx or graphics not loaded%
    }{See the gnuplot documentation for explanation.%
    }{The gnuplot epslatex terminal needs graphicx.sty or graphics.sty.}%
    \renewcommand\includegraphics[2][]{}%
  }%
  \providecommand\rotatebox[2]{#2}%
  \@ifundefined{ifGPcolor}{%
    \newif\ifGPcolor
    \GPcolorfalse
  }{}%
  \@ifundefined{ifGPblacktext}{%
    \newif\ifGPblacktext
    \GPblacktexttrue
  }{}%
  % define a \g@addto@macro without @ in the name:
  \let\gplgaddtomacro\g@addto@macro
  % define empty templates for all commands taking text:
  \gdef\gplbacktext{}%
  \gdef\gplfronttext{}%
  \makeatother
  \ifGPblacktext
    % no textcolor at all
    \def\colorrgb#1{}%
    \def\colorgray#1{}%
  \else
    % gray or color?
    \ifGPcolor
      \def\colorrgb#1{\color[rgb]{#1}}%
      \def\colorgray#1{\color[gray]{#1}}%
      \expandafter\def\csname LTw\endcsname{\color{white}}%
      \expandafter\def\csname LTb\endcsname{\color{black}}%
      \expandafter\def\csname LTa\endcsname{\color{black}}%
      \expandafter\def\csname LT0\endcsname{\color[rgb]{1,0,0}}%
      \expandafter\def\csname LT1\endcsname{\color[rgb]{0,1,0}}%
      \expandafter\def\csname LT2\endcsname{\color[rgb]{0,0,1}}%
      \expandafter\def\csname LT3\endcsname{\color[rgb]{1,0,1}}%
      \expandafter\def\csname LT4\endcsname{\color[rgb]{0,1,1}}%
      \expandafter\def\csname LT5\endcsname{\color[rgb]{1,1,0}}%
      \expandafter\def\csname LT6\endcsname{\color[rgb]{0,0,0}}%
      \expandafter\def\csname LT7\endcsname{\color[rgb]{1,0.3,0}}%
      \expandafter\def\csname LT8\endcsname{\color[rgb]{0.5,0.5,0.5}}%
    \else
      % gray
      \def\colorrgb#1{\color{black}}%
      \def\colorgray#1{\color[gray]{#1}}%
      \expandafter\def\csname LTw\endcsname{\color{white}}%
      \expandafter\def\csname LTb\endcsname{\color{black}}%
      \expandafter\def\csname LTa\endcsname{\color{black}}%
      \expandafter\def\csname LT0\endcsname{\color{black}}%
      \expandafter\def\csname LT1\endcsname{\color{black}}%
      \expandafter\def\csname LT2\endcsname{\color{black}}%
      \expandafter\def\csname LT3\endcsname{\color{black}}%
      \expandafter\def\csname LT4\endcsname{\color{black}}%
      \expandafter\def\csname LT5\endcsname{\color{black}}%
      \expandafter\def\csname LT6\endcsname{\color{black}}%
      \expandafter\def\csname LT7\endcsname{\color{black}}%
      \expandafter\def\csname LT8\endcsname{\color{black}}%
    \fi
  \fi
  \setlength{\unitlength}{0.0500bp}%
  \begin{picture}(9000.00,4500.00)%
    \gplgaddtomacro\gplbacktext{%
      \colorrgb{0.00,0.00,0.00}%
      \put(740,240){\makebox(0,0)[r]{\strut{}0}}%
      \colorrgb{0.00,0.00,0.00}%
      \put(740,1044){\makebox(0,0)[r]{\strut{}0.2}}%
      \colorrgb{0.00,0.00,0.00}%
      \put(740,1848){\makebox(0,0)[r]{\strut{}0.4}}%
      \colorrgb{0.00,0.00,0.00}%
      \put(740,2651){\makebox(0,0)[r]{\strut{}0.6}}%
      \colorrgb{0.00,0.00,0.00}%
      \put(740,3455){\makebox(0,0)[r]{\strut{}0.8}}%
      \colorrgb{0.00,0.00,0.00}%
      \put(740,4259){\makebox(0,0)[r]{\strut{}1}}%
      \colorrgb{0.00,0.00,0.00}%
      \put(160,2249){\rotatebox{90}{\makebox(0,0){\strut{}Entrop\'ia}}}%
    }%
    \gplgaddtomacro\gplfronttext{%
      \colorrgb{0.00,0.00,0.00}%
      \put(1148,52){\rotatebox{90}{\makebox(0,0)[r]{\strut{}172.17.88.6}}}%
      \put(1436,52){\rotatebox{90}{\makebox(0,0)[r]{\strut{}172.17.14.193}}}%
      \put(1724,52){\rotatebox{90}{\makebox(0,0)[r]{\strut{}172.17.28.173}}}%
      \put(2012,52){\rotatebox{90}{\makebox(0,0)[r]{\strut{}172.17.201.92}}}%
      \put(2301,52){\rotatebox{90}{\makebox(0,0)[r]{\strut{}172.17.8.129}}}%
      \put(2589,52){\rotatebox{90}{\makebox(0,0)[r]{\strut{}172.17.39.172}}}%
      \put(2877,52){\rotatebox{90}{\makebox(0,0)[r]{\strut{}172.17.46.252}}}%
      \put(3165,52){\rotatebox{90}{\makebox(0,0)[r]{\strut{}172.17.58.212}}}%
      \put(3453,52){\rotatebox{90}{\makebox(0,0)[r]{\strut{}172.17.0.1}}}%
      \put(3741,52){\rotatebox{90}{\makebox(0,0)[r]{\strut{}172.17.180.136}}}%
      \put(4029,52){\rotatebox{90}{\makebox(0,0)[r]{\strut{}172.17.52.76}}}%
      \put(4317,52){\rotatebox{90}{\makebox(0,0)[r]{\strut{}172.17.61.73}}}%
      \put(4605,52){\rotatebox{90}{\makebox(0,0)[r]{\strut{}172.17.41.86}}}%
      \put(4894,52){\rotatebox{90}{\makebox(0,0)[r]{\strut{}172.17.11.60}}}%
      \put(5182,52){\rotatebox{90}{\makebox(0,0)[r]{\strut{}172.17.74.1}}}%
      \put(5470,52){\rotatebox{90}{\makebox(0,0)[r]{\strut{}172.17.160.48}}}%
      \put(5758,52){\rotatebox{90}{\makebox(0,0)[r]{\strut{}172.17.9.121}}}%
      \put(6046,52){\rotatebox{90}{\makebox(0,0)[r]{\strut{}172.17.163.67}}}%
      \put(6334,52){\rotatebox{90}{\makebox(0,0)[r]{\strut{}172.17.84.225}}}%
      \put(6622,52){\rotatebox{90}{\makebox(0,0)[r]{\strut{}172.17.215.194}}}%
      \put(6910,52){\rotatebox{90}{\makebox(0,0)[r]{\strut{}172.17.167.112}}}%
      \put(7198,52){\rotatebox{90}{\makebox(0,0)[r]{\strut{}172.17.77.172}}}%
      \put(7487,52){\rotatebox{90}{\makebox(0,0)[r]{\strut{}172.17.77.135}}}%
      \put(7775,52){\rotatebox{90}{\makebox(0,0)[r]{\strut{}172.17.67.120}}}%
      \put(8063,52){\rotatebox{90}{\makebox(0,0)[r]{\strut{}172.17.60.76}}}%
      \put(8351,52){\rotatebox{90}{\makebox(0,0)[r]{\strut{}172.17.76.210}}}%
    }%
    \gplbacktext
    \put(0,0){\includegraphics{altopalermo-destination}}%
    \gplfronttext
  \end{picture}%
\endgroup



\subsection{Red \emph{McDonald's}}

\grafo{mcdonalds}

\grafo{mcdonalds-172.17.12.2}

\grafo{mcdonalds-172.17.12.1}

\grafo{mcdonalds-172.17.203.1}

\grafo{mcdonalds-0.0.0.0}


\subsubsection{Direcciones IP origen de paquetes ARP \emph{who-has} modeladas como fuentes de información}

% GNUPLOT: LaTeX picture with Postscript
\begingroup
  \makeatletter
  \providecommand\color[2][]{%
    \GenericError{(gnuplot) \space\space\space\@spaces}{%
      Package color not loaded in conjunction with
      terminal option `colourtext'%
    }{See the gnuplot documentation for explanation.%
    }{Either use 'blacktext' in gnuplot or load the package
      color.sty in LaTeX.}%
    \renewcommand\color[2][]{}%
  }%
  \providecommand\includegraphics[2][]{%
    \GenericError{(gnuplot) \space\space\space\@spaces}{%
      Package graphicx or graphics not loaded%
    }{See the gnuplot documentation for explanation.%
    }{The gnuplot epslatex terminal needs graphicx.sty or graphics.sty.}%
    \renewcommand\includegraphics[2][]{}%
  }%
  \providecommand\rotatebox[2]{#2}%
  \@ifundefined{ifGPcolor}{%
    \newif\ifGPcolor
    \GPcolorfalse
  }{}%
  \@ifundefined{ifGPblacktext}{%
    \newif\ifGPblacktext
    \GPblacktexttrue
  }{}%
  % define a \g@addto@macro without @ in the name:
  \let\gplgaddtomacro\g@addto@macro
  % define empty templates for all commands taking text:
  \gdef\gplbacktext{}%
  \gdef\gplfronttext{}%
  \makeatother
  \ifGPblacktext
    % no textcolor at all
    \def\colorrgb#1{}%
    \def\colorgray#1{}%
  \else
    % gray or color?
    \ifGPcolor
      \def\colorrgb#1{\color[rgb]{#1}}%
      \def\colorgray#1{\color[gray]{#1}}%
      \expandafter\def\csname LTw\endcsname{\color{white}}%
      \expandafter\def\csname LTb\endcsname{\color{black}}%
      \expandafter\def\csname LTa\endcsname{\color{black}}%
      \expandafter\def\csname LT0\endcsname{\color[rgb]{1,0,0}}%
      \expandafter\def\csname LT1\endcsname{\color[rgb]{0,1,0}}%
      \expandafter\def\csname LT2\endcsname{\color[rgb]{0,0,1}}%
      \expandafter\def\csname LT3\endcsname{\color[rgb]{1,0,1}}%
      \expandafter\def\csname LT4\endcsname{\color[rgb]{0,1,1}}%
      \expandafter\def\csname LT5\endcsname{\color[rgb]{1,1,0}}%
      \expandafter\def\csname LT6\endcsname{\color[rgb]{0,0,0}}%
      \expandafter\def\csname LT7\endcsname{\color[rgb]{1,0.3,0}}%
      \expandafter\def\csname LT8\endcsname{\color[rgb]{0.5,0.5,0.5}}%
    \else
      % gray
      \def\colorrgb#1{\color{black}}%
      \def\colorgray#1{\color[gray]{#1}}%
      \expandafter\def\csname LTw\endcsname{\color{white}}%
      \expandafter\def\csname LTb\endcsname{\color{black}}%
      \expandafter\def\csname LTa\endcsname{\color{black}}%
      \expandafter\def\csname LT0\endcsname{\color{black}}%
      \expandafter\def\csname LT1\endcsname{\color{black}}%
      \expandafter\def\csname LT2\endcsname{\color{black}}%
      \expandafter\def\csname LT3\endcsname{\color{black}}%
      \expandafter\def\csname LT4\endcsname{\color{black}}%
      \expandafter\def\csname LT5\endcsname{\color{black}}%
      \expandafter\def\csname LT6\endcsname{\color{black}}%
      \expandafter\def\csname LT7\endcsname{\color{black}}%
      \expandafter\def\csname LT8\endcsname{\color{black}}%
    \fi
  \fi
  \setlength{\unitlength}{0.0500bp}%
  \begin{picture}(9000.00,4500.00)%
    \gplgaddtomacro\gplbacktext{%
      \colorrgb{0.00,0.00,0.00}%
      \put(500,240){\makebox(0,0)[r]{\strut{}0}}%
      \colorrgb{0.00,0.00,0.00}%
      \put(500,910){\makebox(0,0)[r]{\strut{}1}}%
      \colorrgb{0.00,0.00,0.00}%
      \put(500,1580){\makebox(0,0)[r]{\strut{}2}}%
      \colorrgb{0.00,0.00,0.00}%
      \put(500,2250){\makebox(0,0)[r]{\strut{}3}}%
      \colorrgb{0.00,0.00,0.00}%
      \put(500,2919){\makebox(0,0)[r]{\strut{}4}}%
      \colorrgb{0.00,0.00,0.00}%
      \put(500,3589){\makebox(0,0)[r]{\strut{}5}}%
      \colorrgb{0.00,0.00,0.00}%
      \put(500,4259){\makebox(0,0)[r]{\strut{}6}}%
      \colorrgb{0.00,0.00,0.00}%
      \put(160,2249){\rotatebox{90}{\makebox(0,0){\strut{}Entrop\'ia}}}%
    }%
    \gplgaddtomacro\gplfronttext{%
      \colorrgb{0.00,0.00,0.00}%
      \put(620,52){\rotatebox{90}{\makebox(0,0)[r]{\strut{}172.17.12.247}}}%
      \put(731,52){\rotatebox{90}{\makebox(0,0)[r]{\strut{}172.17.12.245}}}%
      \put(843,52){\rotatebox{90}{\makebox(0,0)[r]{\strut{}172.17.203.111}}}%
      \put(954,52){\rotatebox{90}{\makebox(0,0)[r]{\strut{}172.17.12.212}}}%
      \put(1066,52){\rotatebox{90}{\makebox(0,0)[r]{\strut{}172.17.12.38}}}%
      \put(1177,52){\rotatebox{90}{\makebox(0,0)[r]{\strut{}172.17.12.36}}}%
      \put(1288,52){\rotatebox{90}{\makebox(0,0)[r]{\strut{}10.173.104.158}}}%
      \put(1400,52){\rotatebox{90}{\makebox(0,0)[r]{\strut{}172.17.12.33}}}%
      \put(1511,52){\rotatebox{90}{\makebox(0,0)[r]{\strut{}172.17.12.31}}}%
      \put(1622,52){\rotatebox{90}{\makebox(0,0)[r]{\strut{}169.254.14.100}}}%
      \put(1734,52){\rotatebox{90}{\makebox(0,0)[r]{\strut{}172.17.12.14}}}%
      \put(1845,52){\rotatebox{90}{\makebox(0,0)[r]{\strut{}172.17.12.15}}}%
      \put(1957,52){\rotatebox{90}{\makebox(0,0)[r]{\strut{}172.17.12.73}}}%
      \put(2068,52){\rotatebox{90}{\makebox(0,0)[r]{\strut{}172.17.12.56}}}%
      \put(2179,52){\rotatebox{90}{\makebox(0,0)[r]{\strut{}172.17.12.1}}}%
      \put(2291,52){\rotatebox{90}{\makebox(0,0)[r]{\strut{}10.123.44.74}}}%
      \put(2402,52){\rotatebox{90}{\makebox(0,0)[r]{\strut{}172.17.12.4}}}%
      \put(2513,52){\rotatebox{90}{\makebox(0,0)[r]{\strut{}172.17.12.5}}}%
      \put(2625,52){\rotatebox{90}{\makebox(0,0)[r]{\strut{}169.254.159.66}}}%
      \put(2736,52){\rotatebox{90}{\makebox(0,0)[r]{\strut{}172.17.12.254}}}%
      \put(2848,52){\rotatebox{90}{\makebox(0,0)[r]{\strut{}172.17.12.253}}}%
      \put(2959,52){\rotatebox{90}{\makebox(0,0)[r]{\strut{}172.17.12.28}}}%
      \put(3070,52){\rotatebox{90}{\makebox(0,0)[r]{\strut{}172.17.12.13}}}%
      \put(3182,52){\rotatebox{90}{\makebox(0,0)[r]{\strut{}169.254.84.225}}}%
      \put(3293,52){\rotatebox{90}{\makebox(0,0)[r]{\strut{}172.17.203.107}}}%
      \put(3404,52){\rotatebox{90}{\makebox(0,0)[r]{\strut{}172.17.203.191}}}%
      \put(3516,52){\rotatebox{90}{\makebox(0,0)[r]{\strut{}172.17.12.20}}}%
      \put(3627,52){\rotatebox{90}{\makebox(0,0)[r]{\strut{}169.254.163.81}}}%
      \put(3739,52){\rotatebox{90}{\makebox(0,0)[r]{\strut{}169.254.79.68}}}%
      \put(3850,52){\rotatebox{90}{\makebox(0,0)[r]{\strut{}10.109.46.225}}}%
      \put(3961,52){\rotatebox{90}{\makebox(0,0)[r]{\strut{}100.108.113.237}}}%
      \put(4073,52){\rotatebox{90}{\makebox(0,0)[r]{\strut{}172.17.203.156}}}%
      \put(4184,52){\rotatebox{90}{\makebox(0,0)[r]{\strut{}172.17.12.40}}}%
      \put(4295,52){\rotatebox{90}{\makebox(0,0)[r]{\strut{}172.17.12.47}}}%
      \put(4407,52){\rotatebox{90}{\makebox(0,0)[r]{\strut{}172.17.12.35}}}%
      \put(4518,52){\rotatebox{90}{\makebox(0,0)[r]{\strut{}10.123.35.127}}}%
      \put(4630,52){\rotatebox{90}{\makebox(0,0)[r]{\strut{}100.75.141.85}}}%
      \put(4741,52){\rotatebox{90}{\makebox(0,0)[r]{\strut{}172.17.12.183}}}%
      \put(4852,52){\rotatebox{90}{\makebox(0,0)[r]{\strut{}172.17.12.200}}}%
      \put(4964,52){\rotatebox{90}{\makebox(0,0)[r]{\strut{}172.17.12.248}}}%
      \put(5075,52){\rotatebox{90}{\makebox(0,0)[r]{\strut{}172.17.12.239}}}%
      \put(5186,52){\rotatebox{90}{\makebox(0,0)[r]{\strut{}172.17.12.8}}}%
      \put(5298,52){\rotatebox{90}{\makebox(0,0)[r]{\strut{}172.17.12.128}}}%
      \put(5409,52){\rotatebox{90}{\makebox(0,0)[r]{\strut{}172.17.12.246}}}%
      \put(5521,52){\rotatebox{90}{\makebox(0,0)[r]{\strut{}172.17.12.240}}}%
      \put(5632,52){\rotatebox{90}{\makebox(0,0)[r]{\strut{}172.17.12.12}}}%
      \put(5743,52){\rotatebox{90}{\makebox(0,0)[r]{\strut{}172.17.12.17}}}%
      \put(5855,52){\rotatebox{90}{\makebox(0,0)[r]{\strut{}172.17.12.81}}}%
      \put(5966,52){\rotatebox{90}{\makebox(0,0)[r]{\strut{}172.17.12.189}}}%
      \put(6077,52){\rotatebox{90}{\makebox(0,0)[r]{\strut{}172.17.12.43}}}%
      \put(6189,52){\rotatebox{90}{\makebox(0,0)[r]{\strut{}172.17.12.3}}}%
      \put(6300,52){\rotatebox{90}{\makebox(0,0)[r]{\strut{}172.17.12.205}}}%
      \put(6412,52){\rotatebox{90}{\makebox(0,0)[r]{\strut{}172.17.12.11}}}%
      \put(6523,52){\rotatebox{90}{\makebox(0,0)[r]{\strut{}172.17.12.146}}}%
      \put(6634,52){\rotatebox{90}{\makebox(0,0)[r]{\strut{}172.17.12.251}}}%
      \put(6746,52){\rotatebox{90}{\makebox(0,0)[r]{\strut{}172.17.12.250}}}%
      \put(6857,52){\rotatebox{90}{\makebox(0,0)[r]{\strut{}172.17.12.19}}}%
      \put(6968,52){\rotatebox{90}{\makebox(0,0)[r]{\strut{}172.17.12.241}}}%
      \put(7080,52){\rotatebox{90}{\makebox(0,0)[r]{\strut{}172.17.12.16}}}%
      \put(7191,52){\rotatebox{90}{\makebox(0,0)[r]{\strut{}172.17.12.27}}}%
      \put(7303,52){\rotatebox{90}{\makebox(0,0)[r]{\strut{}172.17.12.203}}}%
      \put(7414,52){\rotatebox{90}{\makebox(0,0)[r]{\strut{}172.17.12.32}}}%
      \put(7525,52){\rotatebox{90}{\makebox(0,0)[r]{\strut{}172.17.12.220}}}%
      \put(7637,52){\rotatebox{90}{\makebox(0,0)[r]{\strut{}172.17.12.22}}}%
      \put(7748,52){\rotatebox{90}{\makebox(0,0)[r]{\strut{}172.17.12.237}}}%
      \put(7859,52){\rotatebox{90}{\makebox(0,0)[r]{\strut{}172.17.12.29}}}%
      \put(7971,52){\rotatebox{90}{\makebox(0,0)[r]{\strut{}172.17.12.121}}}%
      \put(8082,52){\rotatebox{90}{\makebox(0,0)[r]{\strut{}172.17.12.34}}}%
      \put(8194,52){\rotatebox{90}{\makebox(0,0)[r]{\strut{}172.17.12.219}}}%
      \put(8305,52){\rotatebox{90}{\makebox(0,0)[r]{\strut{}172.17.12.24}}}%
      \put(8416,52){\rotatebox{90}{\makebox(0,0)[r]{\strut{}172.17.12.215}}}%
      \put(8528,52){\rotatebox{90}{\makebox(0,0)[r]{\strut{}0.0.0.0}}}%
      \put(8639,52){\rotatebox{90}{\makebox(0,0)[r]{\strut{}172.17.12.2}}}%
    }%
    \gplbacktext
    \put(0,0){\includegraphics{mcdonalds-source}}%
    \gplfronttext
  \end{picture}%
\endgroup



\subsubsection{Direcciones IP destino de paquetes ARP \emph{who-has} modeladas como fuentes de información}
% GNUPLOT: LaTeX picture with Postscript
\begingroup
  \makeatletter
  \providecommand\color[2][]{%
    \GenericError{(gnuplot) \space\space\space\@spaces}{%
      Package color not loaded in conjunction with
      terminal option `colourtext'%
    }{See the gnuplot documentation for explanation.%
    }{Either use 'blacktext' in gnuplot or load the package
      color.sty in LaTeX.}%
    \renewcommand\color[2][]{}%
  }%
  \providecommand\includegraphics[2][]{%
    \GenericError{(gnuplot) \space\space\space\@spaces}{%
      Package graphicx or graphics not loaded%
    }{See the gnuplot documentation for explanation.%
    }{The gnuplot epslatex terminal needs graphicx.sty or graphics.sty.}%
    \renewcommand\includegraphics[2][]{}%
  }%
  \providecommand\rotatebox[2]{#2}%
  \@ifundefined{ifGPcolor}{%
    \newif\ifGPcolor
    \GPcolorfalse
  }{}%
  \@ifundefined{ifGPblacktext}{%
    \newif\ifGPblacktext
    \GPblacktexttrue
  }{}%
  % define a \g@addto@macro without @ in the name:
  \let\gplgaddtomacro\g@addto@macro
  % define empty templates for all commands taking text:
  \gdef\gplbacktext{}%
  \gdef\gplfronttext{}%
  \makeatother
  \ifGPblacktext
    % no textcolor at all
    \def\colorrgb#1{}%
    \def\colorgray#1{}%
  \else
    % gray or color?
    \ifGPcolor
      \def\colorrgb#1{\color[rgb]{#1}}%
      \def\colorgray#1{\color[gray]{#1}}%
      \expandafter\def\csname LTw\endcsname{\color{white}}%
      \expandafter\def\csname LTb\endcsname{\color{black}}%
      \expandafter\def\csname LTa\endcsname{\color{black}}%
      \expandafter\def\csname LT0\endcsname{\color[rgb]{1,0,0}}%
      \expandafter\def\csname LT1\endcsname{\color[rgb]{0,1,0}}%
      \expandafter\def\csname LT2\endcsname{\color[rgb]{0,0,1}}%
      \expandafter\def\csname LT3\endcsname{\color[rgb]{1,0,1}}%
      \expandafter\def\csname LT4\endcsname{\color[rgb]{0,1,1}}%
      \expandafter\def\csname LT5\endcsname{\color[rgb]{1,1,0}}%
      \expandafter\def\csname LT6\endcsname{\color[rgb]{0,0,0}}%
      \expandafter\def\csname LT7\endcsname{\color[rgb]{1,0.3,0}}%
      \expandafter\def\csname LT8\endcsname{\color[rgb]{0.5,0.5,0.5}}%
    \else
      % gray
      \def\colorrgb#1{\color{black}}%
      \def\colorgray#1{\color[gray]{#1}}%
      \expandafter\def\csname LTw\endcsname{\color{white}}%
      \expandafter\def\csname LTb\endcsname{\color{black}}%
      \expandafter\def\csname LTa\endcsname{\color{black}}%
      \expandafter\def\csname LT0\endcsname{\color{black}}%
      \expandafter\def\csname LT1\endcsname{\color{black}}%
      \expandafter\def\csname LT2\endcsname{\color{black}}%
      \expandafter\def\csname LT3\endcsname{\color{black}}%
      \expandafter\def\csname LT4\endcsname{\color{black}}%
      \expandafter\def\csname LT5\endcsname{\color{black}}%
      \expandafter\def\csname LT6\endcsname{\color{black}}%
      \expandafter\def\csname LT7\endcsname{\color{black}}%
      \expandafter\def\csname LT8\endcsname{\color{black}}%
    \fi
  \fi
  \setlength{\unitlength}{0.0500bp}%
  \begin{picture}(9000.00,4500.00)%
    \gplgaddtomacro\gplbacktext{%
      \colorrgb{0.00,0.00,0.00}%
      \put(500,240){\makebox(0,0)[r]{\strut{}0}}%
      \colorrgb{0.00,0.00,0.00}%
      \put(500,1044){\makebox(0,0)[r]{\strut{}1}}%
      \colorrgb{0.00,0.00,0.00}%
      \put(500,1848){\makebox(0,0)[r]{\strut{}2}}%
      \colorrgb{0.00,0.00,0.00}%
      \put(500,2651){\makebox(0,0)[r]{\strut{}3}}%
      \colorrgb{0.00,0.00,0.00}%
      \put(500,3455){\makebox(0,0)[r]{\strut{}4}}%
      \colorrgb{0.00,0.00,0.00}%
      \put(500,4259){\makebox(0,0)[r]{\strut{}5}}%
      \colorrgb{0.00,0.00,0.00}%
      \put(160,2249){\rotatebox{90}{\makebox(0,0){\strut{}Entrop\'ia}}}%
    }%
    \gplgaddtomacro\gplfronttext{%
      \colorrgb{0.00,0.00,0.00}%
      \put(620,52){\rotatebox{90}{\makebox(0,0)[r]{\strut{}172.17.12.56}}}%
      \put(710,52){\rotatebox{90}{\makebox(0,0)[r]{\strut{}172.17.12.189}}}%
      \put(800,52){\rotatebox{90}{\makebox(0,0)[r]{\strut{}172.17.12.48}}}%
      \put(890,52){\rotatebox{90}{\makebox(0,0)[r]{\strut{}172.17.12.47}}}%
      \put(980,52){\rotatebox{90}{\makebox(0,0)[r]{\strut{}172.17.12.45}}}%
      \put(1071,52){\rotatebox{90}{\makebox(0,0)[r]{\strut{}172.17.12.44}}}%
      \put(1161,52){\rotatebox{90}{\makebox(0,0)[r]{\strut{}172.17.12.42}}}%
      \put(1251,52){\rotatebox{90}{\makebox(0,0)[r]{\strut{}172.17.12.41}}}%
      \put(1341,52){\rotatebox{90}{\makebox(0,0)[r]{\strut{}172.17.12.40}}}%
      \put(1431,52){\rotatebox{90}{\makebox(0,0)[r]{\strut{}172.17.12.203}}}%
      \put(1521,52){\rotatebox{90}{\makebox(0,0)[r]{\strut{}169.254.241.64}}}%
      \put(1611,52){\rotatebox{90}{\makebox(0,0)[r]{\strut{}50.22.225.66}}}%
      \put(1701,52){\rotatebox{90}{\makebox(0,0)[r]{\strut{}172.17.12.8}}}%
      \put(1791,52){\rotatebox{90}{\makebox(0,0)[r]{\strut{}172.17.12.9}}}%
      \put(1881,52){\rotatebox{90}{\makebox(0,0)[r]{\strut{}172.17.12.6}}}%
      \put(1972,52){\rotatebox{90}{\makebox(0,0)[r]{\strut{}172.17.12.7}}}%
      \put(2062,52){\rotatebox{90}{\makebox(0,0)[r]{\strut{}172.17.12.5}}}%
      \put(2152,52){\rotatebox{90}{\makebox(0,0)[r]{\strut{}172.17.12.3}}}%
      \put(2242,52){\rotatebox{90}{\makebox(0,0)[r]{\strut{}173.194.42.35}}}%
      \put(2332,52){\rotatebox{90}{\makebox(0,0)[r]{\strut{}172.17.203.2}}}%
      \put(2422,52){\rotatebox{90}{\makebox(0,0)[r]{\strut{}74.125.21.95}}}%
      \put(2512,52){\rotatebox{90}{\makebox(0,0)[r]{\strut{}54.207.65.66}}}%
      \put(2602,52){\rotatebox{90}{\makebox(0,0)[r]{\strut{}173.192.222.176}}}%
      \put(2692,52){\rotatebox{90}{\makebox(0,0)[r]{\strut{}169.254.144.159}}}%
      \put(2782,52){\rotatebox{90}{\makebox(0,0)[r]{\strut{}69.171.235.48}}}%
      \put(2873,52){\rotatebox{90}{\makebox(0,0)[r]{\strut{}172.17.12.10}}}%
      \put(2963,52){\rotatebox{90}{\makebox(0,0)[r]{\strut{}172.17.12.11}}}%
      \put(3053,52){\rotatebox{90}{\makebox(0,0)[r]{\strut{}172.17.12.12}}}%
      \put(3143,52){\rotatebox{90}{\makebox(0,0)[r]{\strut{}172.17.12.13}}}%
      \put(3233,52){\rotatebox{90}{\makebox(0,0)[r]{\strut{}172.17.12.14}}}%
      \put(3323,52){\rotatebox{90}{\makebox(0,0)[r]{\strut{}172.17.12.15}}}%
      \put(3413,52){\rotatebox{90}{\makebox(0,0)[r]{\strut{}172.17.12.17}}}%
      \put(3503,52){\rotatebox{90}{\makebox(0,0)[r]{\strut{}172.17.12.18}}}%
      \put(3593,52){\rotatebox{90}{\makebox(0,0)[r]{\strut{}172.17.12.19}}}%
      \put(3683,52){\rotatebox{90}{\makebox(0,0)[r]{\strut{}169.254.54.153}}}%
      \put(3774,52){\rotatebox{90}{\makebox(0,0)[r]{\strut{}172.17.12.239}}}%
      \put(3864,52){\rotatebox{90}{\makebox(0,0)[r]{\strut{}172.17.12.237}}}%
      \put(3954,52){\rotatebox{90}{\makebox(0,0)[r]{\strut{}172.17.12.128}}}%
      \put(4044,52){\rotatebox{90}{\makebox(0,0)[r]{\strut{}172.17.12.121}}}%
      \put(4134,52){\rotatebox{90}{\makebox(0,0)[r]{\strut{}172.17.12.254}}}%
      \put(4224,52){\rotatebox{90}{\makebox(0,0)[r]{\strut{}172.17.12.250}}}%
      \put(4314,52){\rotatebox{90}{\makebox(0,0)[r]{\strut{}172.17.12.253}}}%
      \put(4404,52){\rotatebox{90}{\makebox(0,0)[r]{\strut{}172.17.12.252}}}%
      \put(4494,52){\rotatebox{90}{\makebox(0,0)[r]{\strut{}172.17.12.220}}}%
      \put(4584,52){\rotatebox{90}{\makebox(0,0)[r]{\strut{}172.17.12.249}}}%
      \put(4675,52){\rotatebox{90}{\makebox(0,0)[r]{\strut{}172.17.12.247}}}%
      \put(4765,52){\rotatebox{90}{\makebox(0,0)[r]{\strut{}172.17.12.244}}}%
      \put(4855,52){\rotatebox{90}{\makebox(0,0)[r]{\strut{}172.17.12.245}}}%
      \put(4945,52){\rotatebox{90}{\makebox(0,0)[r]{\strut{}172.17.12.242}}}%
      \put(5035,52){\rotatebox{90}{\makebox(0,0)[r]{\strut{}172.17.12.243}}}%
      \put(5125,52){\rotatebox{90}{\makebox(0,0)[r]{\strut{}172.17.12.38}}}%
      \put(5215,52){\rotatebox{90}{\makebox(0,0)[r]{\strut{}172.17.12.39}}}%
      \put(5305,52){\rotatebox{90}{\makebox(0,0)[r]{\strut{}172.17.12.36}}}%
      \put(5395,52){\rotatebox{90}{\makebox(0,0)[r]{\strut{}172.17.12.35}}}%
      \put(5485,52){\rotatebox{90}{\makebox(0,0)[r]{\strut{}172.17.12.31}}}%
      \put(5576,52){\rotatebox{90}{\makebox(0,0)[r]{\strut{}172.17.12.28}}}%
      \put(5666,52){\rotatebox{90}{\makebox(0,0)[r]{\strut{}172.17.12.21}}}%
      \put(5756,52){\rotatebox{90}{\makebox(0,0)[r]{\strut{}172.17.12.20}}}%
      \put(5846,52){\rotatebox{90}{\makebox(0,0)[r]{\strut{}172.17.12.23}}}%
      \put(5936,52){\rotatebox{90}{\makebox(0,0)[r]{\strut{}172.17.12.26}}}%
      \put(6026,52){\rotatebox{90}{\makebox(0,0)[r]{\strut{}172.17.12.81}}}%
      \put(6116,52){\rotatebox{90}{\makebox(0,0)[r]{\strut{}172.17.12.46}}}%
      \put(6206,52){\rotatebox{90}{\makebox(0,0)[r]{\strut{}172.17.12.251}}}%
      \put(6296,52){\rotatebox{90}{\makebox(0,0)[r]{\strut{}169.254.79.68}}}%
      \put(6386,52){\rotatebox{90}{\makebox(0,0)[r]{\strut{}172.17.203.1}}}%
      \put(6477,52){\rotatebox{90}{\makebox(0,0)[r]{\strut{}172.17.12.219}}}%
      \put(6567,52){\rotatebox{90}{\makebox(0,0)[r]{\strut{}172.17.12.94}}}%
      \put(6657,52){\rotatebox{90}{\makebox(0,0)[r]{\strut{}172.17.12.240}}}%
      \put(6747,52){\rotatebox{90}{\makebox(0,0)[r]{\strut{}172.17.12.37}}}%
      \put(6837,52){\rotatebox{90}{\makebox(0,0)[r]{\strut{}172.17.12.248}}}%
      \put(6927,52){\rotatebox{90}{\makebox(0,0)[r]{\strut{}169.254.163.81}}}%
      \put(7017,52){\rotatebox{90}{\makebox(0,0)[r]{\strut{}172.17.12.4}}}%
      \put(7107,52){\rotatebox{90}{\makebox(0,0)[r]{\strut{}172.17.12.25}}}%
      \put(7197,52){\rotatebox{90}{\makebox(0,0)[r]{\strut{}169.254.159.66}}}%
      \put(7287,52){\rotatebox{90}{\makebox(0,0)[r]{\strut{}169.254.14.100}}}%
      \put(7378,52){\rotatebox{90}{\makebox(0,0)[r]{\strut{}169.254.84.225}}}%
      \put(7468,52){\rotatebox{90}{\makebox(0,0)[r]{\strut{}172.17.12.24}}}%
      \put(7558,52){\rotatebox{90}{\makebox(0,0)[r]{\strut{}172.17.12.241}}}%
      \put(7648,52){\rotatebox{90}{\makebox(0,0)[r]{\strut{}172.17.12.246}}}%
      \put(7738,52){\rotatebox{90}{\makebox(0,0)[r]{\strut{}172.17.12.43}}}%
      \put(7828,52){\rotatebox{90}{\makebox(0,0)[r]{\strut{}172.17.12.16}}}%
      \put(7918,52){\rotatebox{90}{\makebox(0,0)[r]{\strut{}172.17.12.32}}}%
      \put(8008,52){\rotatebox{90}{\makebox(0,0)[r]{\strut{}172.17.12.29}}}%
      \put(8098,52){\rotatebox{90}{\makebox(0,0)[r]{\strut{}172.17.12.22}}}%
      \put(8188,52){\rotatebox{90}{\makebox(0,0)[r]{\strut{}172.17.12.27}}}%
      \put(8279,52){\rotatebox{90}{\makebox(0,0)[r]{\strut{}172.17.12.34}}}%
      \put(8369,52){\rotatebox{90}{\makebox(0,0)[r]{\strut{}172.17.12.215}}}%
      \put(8459,52){\rotatebox{90}{\makebox(0,0)[r]{\strut{}169.254.255.255}}}%
      \put(8549,52){\rotatebox{90}{\makebox(0,0)[r]{\strut{}172.17.12.1}}}%
      \put(8639,52){\rotatebox{90}{\makebox(0,0)[r]{\strut{}172.17.12.2}}}%
    }%
    \gplbacktext
    \put(0,0){\includegraphics{mcdonalds-destination}}%
    \gplfronttext
  \end{picture}%
\endgroup




\subsection{Red \emph{Starbucks}}

\grafo{starbucks}


\subsubsection{Direcciones IP origen de paquetes ARP \emph{who-has} modeladas como fuentes de información}

% GNUPLOT: LaTeX picture with Postscript
\begingroup
  \makeatletter
  \providecommand\color[2][]{%
    \GenericError{(gnuplot) \space\space\space\@spaces}{%
      Package color not loaded in conjunction with
      terminal option `colourtext'%
    }{See the gnuplot documentation for explanation.%
    }{Either use 'blacktext' in gnuplot or load the package
      color.sty in LaTeX.}%
    \renewcommand\color[2][]{}%
  }%
  \providecommand\includegraphics[2][]{%
    \GenericError{(gnuplot) \space\space\space\@spaces}{%
      Package graphicx or graphics not loaded%
    }{See the gnuplot documentation for explanation.%
    }{The gnuplot epslatex terminal needs graphicx.sty or graphics.sty.}%
    \renewcommand\includegraphics[2][]{}%
  }%
  \providecommand\rotatebox[2]{#2}%
  \@ifundefined{ifGPcolor}{%
    \newif\ifGPcolor
    \GPcolorfalse
  }{}%
  \@ifundefined{ifGPblacktext}{%
    \newif\ifGPblacktext
    \GPblacktexttrue
  }{}%
  % define a \g@addto@macro without @ in the name:
  \let\gplgaddtomacro\g@addto@macro
  % define empty templates for all commands taking text:
  \gdef\gplbacktext{}%
  \gdef\gplfronttext{}%
  \makeatother
  \ifGPblacktext
    % no textcolor at all
    \def\colorrgb#1{}%
    \def\colorgray#1{}%
  \else
    % gray or color?
    \ifGPcolor
      \def\colorrgb#1{\color[rgb]{#1}}%
      \def\colorgray#1{\color[gray]{#1}}%
      \expandafter\def\csname LTw\endcsname{\color{white}}%
      \expandafter\def\csname LTb\endcsname{\color{black}}%
      \expandafter\def\csname LTa\endcsname{\color{black}}%
      \expandafter\def\csname LT0\endcsname{\color[rgb]{1,0,0}}%
      \expandafter\def\csname LT1\endcsname{\color[rgb]{0,1,0}}%
      \expandafter\def\csname LT2\endcsname{\color[rgb]{0,0,1}}%
      \expandafter\def\csname LT3\endcsname{\color[rgb]{1,0,1}}%
      \expandafter\def\csname LT4\endcsname{\color[rgb]{0,1,1}}%
      \expandafter\def\csname LT5\endcsname{\color[rgb]{1,1,0}}%
      \expandafter\def\csname LT6\endcsname{\color[rgb]{0,0,0}}%
      \expandafter\def\csname LT7\endcsname{\color[rgb]{1,0.3,0}}%
      \expandafter\def\csname LT8\endcsname{\color[rgb]{0.5,0.5,0.5}}%
    \else
      % gray
      \def\colorrgb#1{\color{black}}%
      \def\colorgray#1{\color[gray]{#1}}%
      \expandafter\def\csname LTw\endcsname{\color{white}}%
      \expandafter\def\csname LTb\endcsname{\color{black}}%
      \expandafter\def\csname LTa\endcsname{\color{black}}%
      \expandafter\def\csname LT0\endcsname{\color{black}}%
      \expandafter\def\csname LT1\endcsname{\color{black}}%
      \expandafter\def\csname LT2\endcsname{\color{black}}%
      \expandafter\def\csname LT3\endcsname{\color{black}}%
      \expandafter\def\csname LT4\endcsname{\color{black}}%
      \expandafter\def\csname LT5\endcsname{\color{black}}%
      \expandafter\def\csname LT6\endcsname{\color{black}}%
      \expandafter\def\csname LT7\endcsname{\color{black}}%
      \expandafter\def\csname LT8\endcsname{\color{black}}%
    \fi
  \fi
  \setlength{\unitlength}{0.0500bp}%
  \begin{picture}(9000.00,4500.00)%
    \gplgaddtomacro\gplbacktext{%
      \colorrgb{0.00,0.00,0.00}%
      \put(500,240){\makebox(0,0)[r]{\strut{}0}}%
      \colorrgb{0.00,0.00,0.00}%
      \put(500,1044){\makebox(0,0)[r]{\strut{}1}}%
      \colorrgb{0.00,0.00,0.00}%
      \put(500,1848){\makebox(0,0)[r]{\strut{}2}}%
      \colorrgb{0.00,0.00,0.00}%
      \put(500,2651){\makebox(0,0)[r]{\strut{}3}}%
      \colorrgb{0.00,0.00,0.00}%
      \put(500,3455){\makebox(0,0)[r]{\strut{}4}}%
      \colorrgb{0.00,0.00,0.00}%
      \put(500,4259){\makebox(0,0)[r]{\strut{}5}}%
      \colorrgb{0.00,0.00,0.00}%
      \put(160,2249){\rotatebox{90}{\makebox(0,0){\strut{}Entrop\'ia}}}%
    }%
    \gplgaddtomacro\gplfronttext{%
      \colorrgb{0.00,0.00,0.00}%
      \put(620,52){\rotatebox{90}{\makebox(0,0)[r]{\strut{}10.254.31.68}}}%
      \put(897,52){\rotatebox{90}{\makebox(0,0)[r]{\strut{}10.254.31.114}}}%
      \put(1173,52){\rotatebox{90}{\makebox(0,0)[r]{\strut{}10.254.31.33}}}%
      \put(1450,52){\rotatebox{90}{\makebox(0,0)[r]{\strut{}10.254.31.19}}}%
      \put(1726,52){\rotatebox{90}{\makebox(0,0)[r]{\strut{}10.254.31.34}}}%
      \put(2003,52){\rotatebox{90}{\makebox(0,0)[r]{\strut{}10.254.31.37}}}%
      \put(2279,52){\rotatebox{90}{\makebox(0,0)[r]{\strut{}10.254.31.36}}}%
      \put(2556,52){\rotatebox{90}{\makebox(0,0)[r]{\strut{}10.254.31.13}}}%
      \put(2832,52){\rotatebox{90}{\makebox(0,0)[r]{\strut{}10.254.31.38}}}%
      \put(3109,52){\rotatebox{90}{\makebox(0,0)[r]{\strut{}10.254.31.14}}}%
      \put(3385,52){\rotatebox{90}{\makebox(0,0)[r]{\strut{}10.254.31.181}}}%
      \put(3662,52){\rotatebox{90}{\makebox(0,0)[r]{\strut{}10.254.31.245}}}%
      \put(3938,52){\rotatebox{90}{\makebox(0,0)[r]{\strut{}10.254.31.243}}}%
      \put(4215,52){\rotatebox{90}{\makebox(0,0)[r]{\strut{}10.254.31.227}}}%
      \put(4491,52){\rotatebox{90}{\makebox(0,0)[r]{\strut{}10.254.31.204}}}%
      \put(4768,52){\rotatebox{90}{\makebox(0,0)[r]{\strut{}10.254.31.167}}}%
      \put(5044,52){\rotatebox{90}{\makebox(0,0)[r]{\strut{}10.254.31.27}}}%
      \put(5321,52){\rotatebox{90}{\makebox(0,0)[r]{\strut{}10.152.95.4}}}%
      \put(5597,52){\rotatebox{90}{\makebox(0,0)[r]{\strut{}10.254.31.20}}}%
      \put(5874,52){\rotatebox{90}{\makebox(0,0)[r]{\strut{}10.254.31.21}}}%
      \put(6150,52){\rotatebox{90}{\makebox(0,0)[r]{\strut{}10.254.31.29}}}%
      \put(6427,52){\rotatebox{90}{\makebox(0,0)[r]{\strut{}10.254.31.149}}}%
      \put(6703,52){\rotatebox{90}{\makebox(0,0)[r]{\strut{}10.254.31.172}}}%
      \put(6980,52){\rotatebox{90}{\makebox(0,0)[r]{\strut{}10.254.31.17}}}%
      \put(7256,52){\rotatebox{90}{\makebox(0,0)[r]{\strut{}10.254.31.42}}}%
      \put(7533,52){\rotatebox{90}{\makebox(0,0)[r]{\strut{}10.254.31.15}}}%
      \put(7809,52){\rotatebox{90}{\makebox(0,0)[r]{\strut{}10.254.31.35}}}%
      \put(8086,52){\rotatebox{90}{\makebox(0,0)[r]{\strut{}10.254.31.170}}}%
      \put(8362,52){\rotatebox{90}{\makebox(0,0)[r]{\strut{}0.0.0.0}}}%
      \put(8639,52){\rotatebox{90}{\makebox(0,0)[r]{\strut{}10.254.31.1}}}%
    }%
    \gplbacktext
    \put(0,0){\includegraphics{starbucks-source}}%
    \gplfronttext
  \end{picture}%
\endgroup



\subsubsection{Direcciones IP destino de paquetes ARP \emph{who-has} modeladas como fuentes de información}
% GNUPLOT: LaTeX picture with Postscript
\begingroup
  \makeatletter
  \providecommand\color[2][]{%
    \GenericError{(gnuplot) \space\space\space\@spaces}{%
      Package color not loaded in conjunction with
      terminal option `colourtext'%
    }{See the gnuplot documentation for explanation.%
    }{Either use 'blacktext' in gnuplot or load the package
      color.sty in LaTeX.}%
    \renewcommand\color[2][]{}%
  }%
  \providecommand\includegraphics[2][]{%
    \GenericError{(gnuplot) \space\space\space\@spaces}{%
      Package graphicx or graphics not loaded%
    }{See the gnuplot documentation for explanation.%
    }{The gnuplot epslatex terminal needs graphicx.sty or graphics.sty.}%
    \renewcommand\includegraphics[2][]{}%
  }%
  \providecommand\rotatebox[2]{#2}%
  \@ifundefined{ifGPcolor}{%
    \newif\ifGPcolor
    \GPcolorfalse
  }{}%
  \@ifundefined{ifGPblacktext}{%
    \newif\ifGPblacktext
    \GPblacktexttrue
  }{}%
  % define a \g@addto@macro without @ in the name:
  \let\gplgaddtomacro\g@addto@macro
  % define empty templates for all commands taking text:
  \gdef\gplbacktext{}%
  \gdef\gplfronttext{}%
  \makeatother
  \ifGPblacktext
    % no textcolor at all
    \def\colorrgb#1{}%
    \def\colorgray#1{}%
  \else
    % gray or color?
    \ifGPcolor
      \def\colorrgb#1{\color[rgb]{#1}}%
      \def\colorgray#1{\color[gray]{#1}}%
      \expandafter\def\csname LTw\endcsname{\color{white}}%
      \expandafter\def\csname LTb\endcsname{\color{black}}%
      \expandafter\def\csname LTa\endcsname{\color{black}}%
      \expandafter\def\csname LT0\endcsname{\color[rgb]{1,0,0}}%
      \expandafter\def\csname LT1\endcsname{\color[rgb]{0,1,0}}%
      \expandafter\def\csname LT2\endcsname{\color[rgb]{0,0,1}}%
      \expandafter\def\csname LT3\endcsname{\color[rgb]{1,0,1}}%
      \expandafter\def\csname LT4\endcsname{\color[rgb]{0,1,1}}%
      \expandafter\def\csname LT5\endcsname{\color[rgb]{1,1,0}}%
      \expandafter\def\csname LT6\endcsname{\color[rgb]{0,0,0}}%
      \expandafter\def\csname LT7\endcsname{\color[rgb]{1,0.3,0}}%
      \expandafter\def\csname LT8\endcsname{\color[rgb]{0.5,0.5,0.5}}%
    \else
      % gray
      \def\colorrgb#1{\color{black}}%
      \def\colorgray#1{\color[gray]{#1}}%
      \expandafter\def\csname LTw\endcsname{\color{white}}%
      \expandafter\def\csname LTb\endcsname{\color{black}}%
      \expandafter\def\csname LTa\endcsname{\color{black}}%
      \expandafter\def\csname LT0\endcsname{\color{black}}%
      \expandafter\def\csname LT1\endcsname{\color{black}}%
      \expandafter\def\csname LT2\endcsname{\color{black}}%
      \expandafter\def\csname LT3\endcsname{\color{black}}%
      \expandafter\def\csname LT4\endcsname{\color{black}}%
      \expandafter\def\csname LT5\endcsname{\color{black}}%
      \expandafter\def\csname LT6\endcsname{\color{black}}%
      \expandafter\def\csname LT7\endcsname{\color{black}}%
      \expandafter\def\csname LT8\endcsname{\color{black}}%
    \fi
  \fi
  \setlength{\unitlength}{0.0500bp}%
  \begin{picture}(9000.00,4500.00)%
    \gplgaddtomacro\gplbacktext{%
      \colorrgb{0.00,0.00,0.00}%
      \put(500,240){\makebox(0,0)[r]{\strut{}0}}%
      \colorrgb{0.00,0.00,0.00}%
      \put(500,1044){\makebox(0,0)[r]{\strut{}1}}%
      \colorrgb{0.00,0.00,0.00}%
      \put(500,1848){\makebox(0,0)[r]{\strut{}2}}%
      \colorrgb{0.00,0.00,0.00}%
      \put(500,2651){\makebox(0,0)[r]{\strut{}3}}%
      \colorrgb{0.00,0.00,0.00}%
      \put(500,3455){\makebox(0,0)[r]{\strut{}4}}%
      \colorrgb{0.00,0.00,0.00}%
      \put(500,4259){\makebox(0,0)[r]{\strut{}5}}%
      \colorrgb{0.00,0.00,0.00}%
      \put(160,2249){\rotatebox{90}{\makebox(0,0){\strut{}Entrop\'ia}}}%
    }%
    \gplgaddtomacro\gplfronttext{%
      \colorrgb{0.00,0.00,0.00}%
      \put(620,52){\rotatebox{90}{\makebox(0,0)[r]{\strut{}10.254.31.44}}}%
      \put(928,52){\rotatebox{90}{\makebox(0,0)[r]{\strut{}10.254.31.41}}}%
      \put(1237,52){\rotatebox{90}{\makebox(0,0)[r]{\strut{}10.254.31.170}}}%
      \put(1545,52){\rotatebox{90}{\makebox(0,0)[r]{\strut{}10.254.31.30}}}%
      \put(1854,52){\rotatebox{90}{\makebox(0,0)[r]{\strut{}10.254.31.19}}}%
      \put(2162,52){\rotatebox{90}{\makebox(0,0)[r]{\strut{}10.254.31.34}}}%
      \put(2471,52){\rotatebox{90}{\makebox(0,0)[r]{\strut{}10.254.31.37}}}%
      \put(2779,52){\rotatebox{90}{\makebox(0,0)[r]{\strut{}10.254.31.36}}}%
      \put(3087,52){\rotatebox{90}{\makebox(0,0)[r]{\strut{}10.254.31.39}}}%
      \put(3396,52){\rotatebox{90}{\makebox(0,0)[r]{\strut{}10.254.31.38}}}%
      \put(3704,52){\rotatebox{90}{\makebox(0,0)[r]{\strut{}10.254.31.181}}}%
      \put(4013,52){\rotatebox{90}{\makebox(0,0)[r]{\strut{}169.254.151.229}}}%
      \put(4321,52){\rotatebox{90}{\makebox(0,0)[r]{\strut{}10.254.31.149}}}%
      \put(4630,52){\rotatebox{90}{\makebox(0,0)[r]{\strut{}10.254.31.26}}}%
      \put(4938,52){\rotatebox{90}{\makebox(0,0)[r]{\strut{}10.254.31.27}}}%
      \put(5246,52){\rotatebox{90}{\makebox(0,0)[r]{\strut{}10.254.31.25}}}%
      \put(5555,52){\rotatebox{90}{\makebox(0,0)[r]{\strut{}10.254.31.20}}}%
      \put(5863,52){\rotatebox{90}{\makebox(0,0)[r]{\strut{}10.254.31.21}}}%
      \put(6172,52){\rotatebox{90}{\makebox(0,0)[r]{\strut{}10.254.31.28}}}%
      \put(6480,52){\rotatebox{90}{\makebox(0,0)[r]{\strut{}10.254.31.29}}}%
      \put(6788,52){\rotatebox{90}{\makebox(0,0)[r]{\strut{}10.254.31.13}}}%
      \put(7097,52){\rotatebox{90}{\makebox(0,0)[r]{\strut{}10.254.31.15}}}%
      \put(7405,52){\rotatebox{90}{\makebox(0,0)[r]{\strut{}10.254.31.35}}}%
      \put(7714,52){\rotatebox{90}{\makebox(0,0)[r]{\strut{}10.254.31.17}}}%
      \put(8022,52){\rotatebox{90}{\makebox(0,0)[r]{\strut{}10.254.31.42}}}%
      \put(8331,52){\rotatebox{90}{\makebox(0,0)[r]{\strut{}169.254.255.255}}}%
      \put(8639,52){\rotatebox{90}{\makebox(0,0)[r]{\strut{}10.254.31.1}}}%
    }%
    \gplbacktext
    \put(0,0){\includegraphics{starbucks-destination}}%
    \gplfronttext
  \end{picture}%
\endgroup



\subsection{Entropías calculadas}

% Constantes con la entropía calculada
\newcommand{\altopalermoSrcEntropy}{0.109542}
\newcommand{\altopalermoDstEntropy}{3.738743}
\newcommand{\mcdonaldsSrcEntropy}{3.976417}
\newcommand{\mcdonaldsDstEntropy}{3.769457}
\newcommand{\starbucksSrcEntropy}{4.142124}
\newcommand{\starbucksDstEntropy}{3.271891}


\begin{center}
\begin{tabular}{|c|c|c|}
\hline
Red & Fuente de información modelada $S$ & Entropía $H(S)$\\
\hline
Alto Palermo & $S_{src}$ & \altopalermoSrcEntropy\\
Alto Palermo & $S_{dst}$ & \altopalermoDstEntropy\\
\hline
McDonald's   & $S_{src}$ & \mcdonaldsSrcEntropy\\
McDonald's   & $S_{dst}$ & \mcdonaldsDstEntropy\\
\hline
Starbucks    & $S_{src}$ & \starbucksSrcEntropy\\
Starbucks    & $S_{dst}$ & \starbucksDstEntropy\\
\hline
\end{tabular}
\end{center}



%%%%%%%%%%%%%%%%%%%%%%%%%%%%%%%%%%%%%%%%%%%%%%%%%%%%%%%%%%%%%%%%%%%%%%%%%%%%%%%
%% Discusión                      			                                 %%
%%%%%%%%%%%%%%%%%%%%%%%%%%%%%%%%%%%%%%%%%%%%%%%%%%%%%%%%%%%%%%%%%%%%%%%%%%%%%%%


\section{Discusión}

\subsection{Datos encontrados}
\subsubsection{Paquete ARP con IP fuente 0.0.0.0}
Podemos detectar en la red de MC Donalds paquetes ARP Who-Has con ip fuente 0.0.0.0.
La causa de estos ips es debido a lo siguiente:
Cuando un cliente se conecta a un red que posee un servidor DHCP y quiere recibir una IP de esta, manda una petición con su ID en forma de broadcast para que lo detecte el servidor. Una vez detectado por el servidor, este manda una o varias ofertas de IP's a ese ID.
El cliente eventualmente podría recibir la oferta, tomar uno de esos IP's y extraer la dirección del router.
Como el servidor realiza la misma operación con todos los demás clientes que pidan una IP, el cliente debe comprobar que la IP que eligió no la tiene otro cliente. Para esto, coloca en el paquete Who-Has su MAC adress, la ip 0.0.0.0 como fuente para evitar confundir las ARP caches en otros hosts. Si el who-has es respuesto, el cliente rechaza el IP elegido.

%Ver como colocar las referencias
%http://www.ietf.org/rfc/rfc2131.txt
%4.4.1


%%%%%%%%%%%%%%%%%%%%%%%%%%%%%%%%%%%%%%%%%%%%%%%%%%%%%%%%%%%%%%%%%%%%%%%%%%%%%%%
%% Conclusión                                                                %%
%%%%%%%%%%%%%%%%%%%%%%%%%%%%%%%%%%%%%%%%%%%%%%%%%%%%%%%%%%%%%%%%%%%%%%%%%%%%%%%


\section{Conclusión}


\end{document}